\documentclass[a4paper]{article}
\usepackage[a4paper,pdftex]{geometry}
\usepackage[english]{babel}
\usepackage{amsmath,amsfonts}
\usepackage[pdftex]{graphicx}
\usepackage{epstopdf}
\usepackage{fancyhdr}
\usepackage{lastpage}
\usepackage{setspace}
\usepackage{xcolor}
\usepackage{hyperref}
\usepackage{url}
\usepackage[all]{xy}
\usepackage[toc,page]{appendix}
\usepackage[T1]{fontenc}   

% Page style
\pagestyle{fancy}

% Page numbering
\lhead{}
\cfoot{}
\rfoot{\thepage}

\author{Chiel Kooijman\\5743028\\\url{Chiel999@gmail.com} \and
Steven Laan\\6036031\\\url{S.Laan@uva.nl} \and
Camiel Verschoor\\10017321\\\url{Verschoor@uva.nl} \and
Auke Wiggers\\6036163\\\url{A.J.Wiggers@uva.nl}}

\title{Collaborative Visual SLAM\\\normalsize Multi-Agent Visual Odometry and SLAM with humanoid robots.\\Project AI (6 EC)\\Artificial Intelligence\\Faculty of Science\\University of Amsterdam}

\begin{document}

%% FRONT PAGE
\thispagestyle{empty}
\begin{center}
\Large\textsc{Collaborative Visual SLAM}\\
\normalsize\textsc{Multi-Agent Visual Odometry and SLAM with humanoid robots.}

\vspace{2cm}

\begin{figure*}[!ht]
\centering
\includegraphics[width=\textwidth]{images/front.jpg}
\end{figure*}

\subsubsection*{A Artificial Intelligence project by Auke J. Wiggers, Camiel R. Verschoor,\\Chiel Kooijman and Steven Laan}
\end{center}

\newpage

% EMPTY PAGE
\thispagestyle{empty}
\mbox{}
\newpage

% OFFICIAL FRONT PAGE
\maketitle
\clearpage

% TABLE OF CONTENTS
\thispagestyle{empty}
\tableofcontents
\clearpage

\section{Introduction}

\section{Related Work}

\section{Theory}

\section{Pipeline}
In this section, the pipeline of our proposed system is described stepwise.

\begin{figure}[!hb]
%\centerline{
%\xymatrix{
%Calibration\ar[rr] & & Feature\ Extraction \ar[rr] & & Feature\ Matching \ar[rr] & & 3D\ map\ reconstruction \ar[rr] 2D\ and\ 3D\ feature\ matching
%User\ar[rr]^{Posts\ message\ on} & & Twitter \ar[dd]^{Message\ is\ extracted\ by}\\
%& &\\
%Recommendation \ar[uu]^{Play\ to} & & System \ar[ll]^{Provides}\\
%}
%}
\caption{Schematic overview of the pipeline of the system}
\label{fig:system}
\end{figure}

\subsection{Calibration}
\subsection{Feature Extraction}
In this section, we concisely describe the feature extraction methods that were used in the system to extract features. In total three feature extractions methods were applied, namely, Binary Robust Invariant Scalable Keypoints (BRISK) \cite{Leutenegger2011}, Oriented FAST and Rotated BRIEF (ORB) \cite{Rublee2011} and Fast Retina Keypoints (FREAK) \cite{Ortiz2012}.

\subsubsection{Binary Robust Invariant Scalable Keypoints}
BRISK relies on an easily configurable circular sampling pattern from which it computes brightness comparisons to form a binary descriptor string. The unique properties, rotation and scale invariance, of BRISK can be useful for a wide spectrum of applications, in particular for tasks with hard real-time constraints or limited computation power: BRISK finally offers the quality of high-end features in such time-demanding applications.

\subsubsection{Oriented FAST and Rotated BRIEF}
[Insert]
\subsubsection{Fast Retina Keypoint}
[Insert]
\subsection{Feature Matching}
FLANN FEATUREMATCHER
\subsection{3D Map reconstruction}
\subsection{2D feature and 3D feature Matching}



\section{Experimental Setup}

\section{Results}

\section{Discussion}

\section{Conclusion}

\bibliographystyle{apalike}
\bibliography{references}

\end{document}
